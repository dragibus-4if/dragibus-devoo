\documentclass[a4paper]{report}

\usepackage[francais]{babel}
\usepackage[utf8x]{inputenc}
\usepackage{amsmath}
\usepackage{graphicx}
\usepackage[colorinlistoftodos]{todonotes}
\usepackage {enumitem}
\usepackage {xcolor}
\usepackage {pgf-umlsd}
\usepackage {hyperref}


\begin{document}

\title{DevOo}
\author{dragibus}

\maketitle

\tableofcontents


\chapter{Capture des besoins}

\section{Modèle du domaine}

\section{Cas d'utilisation}

\subsection{Description textuelle des cas d'utilisation}
~~\\

Description abrégée du cas d’utilisation \textbf{Superviser les tournées}\\

L’utilisateur s’est préalablement identifié sur la plateforme OptiFret, et,
dans le cas d’un superviseur, la fenêtre de visualisation des feuilles de
route est la première fenêtre apparente. Il pourra parcourir l’ensemble des
feuilles de routes depuis une liste déroulante. Pour chaque feuille de route,
l’utilisateur peut voir la liste des livraisons ordonnée (selon un ordre
défini par le système relatif au trajet à parcourir), dont les informations
sont abrégées. Il peut dérouler chaque livraison pour en avoir les
informations détaillées, puis les enrouler ensuite pour retrouver les
informations abrégées.

Il a également accès à differentes informations comme la position GPS du
livreur ou des inforamtions sur livraisons passées (livraison à l’heure ou en
retard, problemes eventuels). \\


Description abrégée du cas d’utilisation \textbf{Modifier feuille de route en
cours de livraison}\\

Au cas où un changement de l’ordre des livraisons soit nécessaire, le
superviseur doit avoir la possibilité de modifier les feuilles de route en
temps réel. Cela comprend qu’il peut changer l’ordre des livraisons, ajouter
ou bien supprimer des livraisons dans une feuille de route précisée. Le
superviseur ne pourra modifier la feuille de route que pour les livraisons non
effectués ni en cours de route. Après chaque modification le système vérifiera
automatiquement l’intégrité de l’ensemble des livraisons pour cette feuille
avant de les rendre visible pour le livreur concerné.  Si l’intégrité des
livraisons ne peut pas être garantie ou bien qu’un changement de date et
d’heure d’autres livraisons est nécessaire, le superviseur est notifié tout de
suite par le système et la modification n’est pas enregistrée. Une fois que le
superviseur a confirmé la notification, il revient à la fenêtre de la
modification, sauf que la fenêtre montrera l’état initial de la feuille de
route. A chaque modification de la feuille de route (c’est à dire à chaque
modification ou suppression d’une livraison), le système recalcule
automatiquement l’itinéraire.\\


Description abrégée du cas d’utilisation \textbf{Modifier feuille de route
avant le début de livraison}\\

Si le superviseur veut supprimer une livraison de l’application OptiFret. Il
sélectionne la livraison à supprimer à partir d’une liste de livraisons et
appuie sur le bouton “supprimer”. Une boîte de dialogue demande alors de
confirmer l’action.  S’il existe une erreur dans les informations d’une
livraison ou à la demande du client, le superviseur selectionne une livraison
et peut modifier l’adresse, la date, l’horaire, le téléphone de contact ou
encore l’intitulé de la livraison (en dehors du cadre du prototype). Dans le
cas d’une modification de la date de livraison un calendrier s’affiche et
permet de saisir la nouvelle date. A chaque modification de la feuille de
route (c’est à dire à chaque modification ou suppression d’une livraison), le
système recalcule automatiquement l’itinéraire.\\


Description abrégée du cas d’utilisation \textbf{Afficher liste des livraisons}\\

Si le superviseur en a besoin, il peut aussi commander l’affichage d’une liste
de toutes les livraisons à partir de la fenêtre de la liste des livraisons par
zône géographique. En bas de cette fenêtre, il trouve un bouton “Afficher
toutes les livraisons” qui lui amènera à la liste complète. Une fois affichée,
le superviseur a la possibilité de chercher une certaine livraison en
utilisant “Ctrl+F” et d’en sélectionner une pour la modifier. S’il clique sur
le bouton “Modifier” de la livraison choisie, le système va lui demander s’il
veut vraiment modifier cette livraison. La confirmation, par conséquent, lui
amènera à la vue de la modification.\\


Description abrégée du cas d’utilisation \textbf{Contacter client}\\

Dans le cas où la génération des feuilles de route de permet pas de placer la
livraison à un client à l’heure prévue initialement, le superviseur contacte
le client afin de le prévenir et d’obtenir une nouvelle date de livraison. Il
récupère les informations clients à partir de la visualisation des
informations de la livraison et peut alors le contacter soit par téléphone,
soit par adresse électronique. Il peut alors replacer la livraison dans la
liste des demandes de livraisons après avoir modifié la date et l’heure
d’arrivée (cas d’utilisation “Modifier livraison”). L’ajout dans la liste des
demandes de livraisons se met à jour automatiquement par le système
OptiFret.\\


\subsection{Diagramme des cas d'utilisation}


\chapter{Conception}

\section{Description détaillée des cas d'utilisation}


\subsection{}
\begin{itemize}[label = \textbullet, font = \color{orange}]
\item \underline{Nom du cas d'utilisation} : 
\item \underline{Périmètre} : prototype OptiFret
\item \underline{Niveau} : but utilisateur
\item \underline{Acteur Principal} : Superviseur
\item \underline{Parties prenantes et intérêts} :
	\begin{itemize}[label = \textbullet, font = \color{blue}]
    \item \underline{Le Superviseur} : 
    \end{itemize}
\item \underline{Préconditions} :
\item \underline{Postconditions} :
\item \underline{Cas Nominal} :
	\begin{enumerate}
    	\item
    \end{enumerate}
\item \underline{Extensions} :
	\begin{enumerate}
    	\item
    \end{enumerate}
\item \underline{Spécifications particulières} :
	\begin{itemize}[label = \textbullet, font = \color{blue}]
    \item
    \end{itemize}
\item \underline{Fréquence d'occurence} : 
\item \underline{Divers} : 
\begin{itemize}[label = \textbullet, font = \color{blue}]
    \item
    \end{itemize}
\end{itemize}


\subsection{Superviser les tournées}
\begin{itemize}[label = \textbullet, font = \color{orange}]
\item \underline{Nom du cas d'utilisation} : Superviser les tournées
\item \underline{Périmètre} : prototype OptiFret
\item \underline{Niveau} : but utilisateur
\item \underline{Acteur Principal} : Superviseur
\item \underline{Parties prenantes et intérêts} :
	\begin{itemize}[label = \textbullet, font = \color{blue}]
    \item \underline{Le Superviseur} : il veut pouvoir visualiser l'état detaillé d'une tournée.
    \item \underline{Le Livreur} : sa position GPS en temps réelle est envoyée au système, est traitée, et affichée en temps réelle.
    \end{itemize}
\item \underline{Préconditions} : L'utilisateur doit être identifié.
\item \underline{Postconditions} : Néant.
\item \underline{Cas Nominal} : 
	\begin{enumerate}
        \item La fenêtre de supervision s'affiche dès le Superviseur identifié.
        \item La visualisation d'une tournée s'affiche (détails des livraisons et carte avec position du livreur).
        \item Le Superviseur peut changer la feuille de route affichée en selectionnant un autre livreur.
        \item En cliquant sur un bandeau de livraison, le Superviseur a accès au détails concernant la livraison.
    \end{enumerate}
\item \underline{Extensions} :
	\begin{enumerate}
    	\item Aucune tournée n'est en cours : le menu déroulant est vide et un message d'erreur s'affiche en lieu et place de la carte.
        
    \end{enumerate}
\item \underline{Spécifications particulières} :
	\begin{itemize}[label = \textbullet, font = \color{blue}]
    \item La carte doit être actualisée fréquemment
    \end{itemize}
\item \underline{Fréquence d'occurence} : très fréquente
\item \underline{Divers} :
	\begin{itemize}[label = \textbullet, font = \color{blue}]
    \item
    \end{itemize}
\end{itemize}

\subsection{Modifier une feuille de route en cours de livraison}
\begin{itemize}[label = \textbullet, font = \color{orange}]
\item \underline{Nom du cas d'utilisation} : Modifier une feuille de route en cours de livraison
\item \underline{Périmètre} : prototype OptiFret
\item \underline{Niveau} : but utilisateur
\item \underline{Acteur Principal} : Superviseur
\item \underline{Parties prenantes et intérêts} : 
	\begin{itemize}[label = \textbullet, font = \color{blue}]
    \item \underline{Le Superviseur} : il veut pouvoir ajouter ou supprimer une livraison, ou bien intervertir deux livraisons, en temps réel.
    \end{itemize}
\item \underline{Préconditions} : L'utilisateur doit être identifié, la tournée doit être calculée
\item \underline{Postconditions} : Néant
\item \underline{Cas Nominal} :
	\begin{enumerate}
    	\item Le Superviseur se retrouve sur la fenêtre d'affichage des tournées en temps réel, une tournée en cours affichée.
        \item Il peut voir les livraisons déja effectuées (en vert), celles ayant eu un problème (en rouge), et celles non encore traitées (en blanc).
        \item Il supprime une livraison en cliquant sur la croix présente sur le côté.
        \item Il insère une nouvelle livraison.
        \item Il change l'ordre des livraisons en faisant un glissé-déposer sur l'une d'elles.
        \item Il change de tournée en sélectionnant une des autres dans la liste déroulante.
        \item Il peut refaire les points 3, 4 et 5 avec la nouvelle tournée.
    \end{enumerate}
\item \underline{Extensions} :
	\begin{enumerate}
    	\item Le Superviseur essaye de supprimer ou de glissé-déposer une livraison déja effectuée (vert) ou ayant eu un problème (rouge)
        	\begin{enumerate}
            	\item Le système refuse le changement, et affiche un message d'erreur.
            \end{enumerate}
    \end{enumerate}
\item \underline{Spécifications particulières} :
	\begin{itemize}[label = \textbullet, font = \color{blue}]
    \item
    \end{itemize}
\item \underline{Fréquence d'occurence} : Quotidienne.
\item \underline{Divers} :
\begin{itemize}[label = \textbullet, font = \color{blue}]
    \item
    \end{itemize}
\end{itemize}

\subsection{Modifier une feuille de route avant le début de livraison}
\begin{itemize}[label = \textbullet, font = \color{orange}]
\item \underline{Nom du cas d'utilisation} : Modifier une feuille de route avant le début de livraison
\item \underline{Périmètre} : prototype OptiFret
\item \underline{Niveau} : but utilisateur
\item \underline{Acteur Principal} : Superviseur
\item \underline{Parties prenantes et intérêts} :
	\begin{itemize}[label = \textbullet, font = \color{blue}]
    \item \underline{Le Superviseur} : il veut modifier ou supprimer une livraison ou bien changer l'ordre des livraisons
    \end{itemize}
\item \underline{Préconditions} : le Superviseur est identifié et authentifié sur le système, la feuille de route existe sur le système
\item \underline{Postconditions} : les modifications sont enregistrés sur le système et accessible pour le Livreur concerné
\item \underline{Cas Nominal} :
	\begin{enumerate}
        \item Le Superviseur se retrouve sur la vue qui affiche les feuilles de routes.
        \item Il en choisit une dans le menu dropdown et le système charge les données de la feuille précisée.
        \item Le Superviseur choisit une des livraisons comprise dans la tournée et la déplace en drag\&drop.
        \item Le système recalcule automatiquement les nouveaux horaires pour les livraisons et les enregistre directement dans la base de données.
        \item Le Superviseur clique sur le bouton "X" d'une livraison pour la supprimer.
        \item Il clique sur le bouton "+" de la liste des livraisons pour ajouter une autre livraison.
        \item Le système affiche une liste des livraison pas encore inclues dans la tournée qui se situant dans la même zône géographique.
        \item Le Superviseur sélectionne une livraison et clique sur le bouton "OK" pour valider.
        \item Il choisit une nouvelle feuille de route dans le menu dropdown et il recommence.
    \end{enumerate}
\item \underline{Extensions} :
	\begin{enumerate}
    	\item
    \end{enumerate}
\item \underline{Spécifications particulières} :
	\begin{itemize}[label = \textbullet, font = \color{blue}]
    \item
    \end{itemize}
\item \underline{Fréquence d'occurence} : potentiellement plusieurs fois par jour par feuille de route
\item \underline{Divers} :
\begin{itemize}[label = \textbullet, font = \color{blue}]
    \item
    \end{itemize}
\end{itemize}



\pagebreak

\section{Diagramme de séquence : Saisir les demandes de livraison et calculer
la feuille de route} ~~\\

\begin{sequencediagram}
	\newthread{name}{:Thread}
\end{sequencediagram}


\end{document}