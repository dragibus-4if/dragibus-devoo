\documentclass[a4paper]{article}

\usepackage[left=2.75cm, right=2.75cm]{geometry}
\usepackage[francais]{babel}
\usepackage[utf8x]{inputenc}
\usepackage{hyperref}
\usepackage{enumitem}
\usepackage{graphicx}
\usepackage{xcolor}
\usepackage{pdflscape}

% pgf-umlsd
\usepackage{tikz}
\usetikzlibrary{arrows, shadows}
\usepackage[underline=true, rounded corners=false]{pgf-umlsd}

\begin{document}

\title{Projet Dev-OO}
\author{H4104 -- Dragibus}

\maketitle
\pagebreak

\tableofcontents
\pagebreak

%\chapter{Capture des besoins}

\section{Modèle du domaine}
{\huge TODO}

\section{Cas d'utilisation}

\subsection{Description textuelle des cas d'utilisation}

Description abrégée du cas d'utilisation
\textbf{Superviser les tournées} \\

L'utilisateur s'est préalablement identifié sur la plateforme OptiFret, et,
dans le cas d'un Superviseur, la fenêtre de visualisation des feuilles de route
est la première fenêtre apparente. Il pourra parcourir l'ensemble des feuilles
de routes depuis une liste déroulante. Pour chaque feuille de route,
l'utilisateur peut voir la liste des livraisons ordonnée (selon un ordre défini
par le système relatif au trajet à parcourir), dont les informations sont
abrégées. Il peut dérouler chaque livraison pour en avoir les informations
détaillées, puis les enrouler ensuite pour retrouver les informations abrégées.

Il a également accès à differentes informations comme la position GPS du
livreur ou des inforamtions sur livraisons passées (livraison à l'heure ou en
retard, problemes eventuels). \\


Description abrégée du cas d'utilisation
\textbf{Modifier feuille de route en cours de livraison} \\

Au cas où un changement de l'ordre des livraisons soit nécessaire, le
Superviseur doit avoir la possibilité de modifier les feuilles de route en
temps réel. Cela comprend qu'il peut changer l'ordre des livraisons, ajouter ou
bien supprimer des livraisons dans une feuille de route précisée. Le
Superviseur ne pourra modifier la feuille de route que pour les livraisons non
effectués ni en cours de route. Après chaque modification le système vérifiera
automatiquement l'intégrité de l'ensemble des livraisons pour cette feuille
avant de les rendre visible pour le livreur concerné.
Si l'intégrité des livraisons ne peut pas être garantie ou bien qu'un
changement de date et d'heure d'autres livraisons est nécessaire, le
Superviseur est notifié tout de suite par le système et la modification n'est
pas enregistrée. Une fois que le Superviseur a confirmé la notification, il
revient à la fenêtre de la modification, sauf que la fenêtre montrera l'état
initial de la feuille de route.
A chaque modification de la feuille de route (c'est à dire à chaque
modification ou suppression d'une livraison), le système recalcule
automatiquement l'itinéraire. \\

Description abrégée du cas d'utilisation
\textbf{Modifier feuille de route avant le début de livraison} \\

Si le Superviseur veut supprimer une livraison de l'application OptiFret. Il
sélectionne la livraison à supprimer à partir d'une liste de livraisons et
appuie sur le bouton "supprimer". Une boîte de dialogue demande alors de
confirmer l'action.
S'il existe une erreur dans les informations d'une livraison ou à la demande du
client, le Superviseur selectionne une livraison et peut modifier l'adresse, la
date, l'horaire, le téléphone de contact ou encore l'intitulé de la livraison
(en dehors du cadre du prototype). Dans le cas d'une modification de la date de
livraison un calendrier s'affiche et permet de saisir la nouvelle date.
A chaque modification de la feuille de route (c'est à dire à chaque
modification ou suppression d'une livraison), le système recalcule
automatiquement l'itinéraire. \\

Description abrégée du cas d'utilisation
\textbf{Afficher liste des livraisons} \\

Si le Superviseur en a besoin, il peut aussi commander l'affichage d'une liste
de toutes les livraisons à partir de la fenêtre de la liste des livraisons par
zône géographique. En bas de cette fenêtre, il trouve un bouton "Afficher
toutes les livraisons" qui lui amènera à la liste complète. Une fois affichée,
le Superviseur a la possibilité de chercher une certaine livraison en utilisant
"Ctrl+F" et d'en sélectionner une pour la modifier. S'il clique sur le bouton
"Modifier" de la livraison choisie, le système va lui demander s'il veut
vraiment modifier cette livraison. La confirmation, par conséquent, lui amènera
à la vue de la modification. \\

Description abrégée du cas d'utilisation
\textbf{Contacter client} \\

Dans le cas où la génération des feuilles de route de permet pas de placer la
livraison à un client à l'heure prévue initialement, le Superviseur contacte le
client afin de le prévenir et d'obtenir une nouvelle date de livraison. Il
récupère les informations clients à partir de la visualisation des informations
de la livraison et peut alors le contacter soit par téléphone, soit par adresse
électronique. Il peut alors replacer la livraison dans la liste des demandes de
livraisons après avoir modifié la date et l'heure d'arrivée (cas d'utilisation
"Modifier livraison"). L'ajout dans la liste des demandes de livraisons se met
à jour automatiquement par le système OptiFret. \\

\subsection{Diagramme des cas d'utilisation}
{\huge TODO}

\section{Description détaillée des cas d'utilisation}

Dans cette partie nous abrégerons le terme "cas d'utilisation" par "CdU".

\subsection{Visionner les tournées}

\begin{itemize}[label = \textbullet, font = \color{orange}]
    \item \underline{Nom du cas d'utilisation} : visionner les tournées
    \item \underline{Périmètre} : prototype OptiFret
    \item \underline{Niveau} : but utilisateur
    \item \underline{Acteur Principal} : Superviseur
    \item \underline{Parties prenantes et intérêts} :
        \begin{itemize}[label = \textbullet, font = \color{blue}]
            \item \underline{Le Superviseur} : il veut pouvoir visualiser
                l'état detaillé d'une tournée.
        \end{itemize}
    \item \underline{Préconditions} : L'utilisateur doit être identifié, le
        plan et la demande de livraisons chargés.
    \item \underline{Postconditions} : Néant.
    \item \underline{Cas Nominal} :
    \begin{enumerate}
        \item La fenêtre de supervision s'affiche dès le Superviseur identifié.
        \item La visualisation d'une tournée s'affiche (détails des livraisons
            et carte avec position du livreur).
        \item Le Superviseur peut changer la feuille de route affichée en
            selectionnant un autre livreur.
        \item En cliquant sur un bandeau de livraison, le Superviseur a accès
            au détails concernant la livraison.
        \item En re-cliquant sur ce bandeau, le Superviseur peut ré-enrouler
            les détails de livraisons pour les masquer.
    \end{enumerate}
    \item \underline{Extensions} :
    \begin{enumerate}
        \item Aucune carte n'a été chargée : un message d'erreur s'affiche en
            lieu et place de la carte.
        \item Aucune tournée n'est en cours : le menu déroulant est vide.
    \end{enumerate}
    \item \underline{Spécifications particulières} :
    \begin{itemize}[label = \textbullet, font = \color{blue}]
        \item La carte doit être actualisée fréquemment
    \end{itemize}
    \item \underline{Fréquence d'occurence} : très fréquente
    \item \underline{Divers} : Néant.
\end{itemize}

\subsection{Modifier une feuille de route avant le début de livraison}

\begin{itemize}[label = \textbullet, font = \color{orange}]
    \item \underline{Nom du cas d'utilisation} : Modifier une feuille de route
        avant le début de livraison
    \item \underline{Périmètre} : prototype OptiFret
    \item \underline{Niveau} : but utilisateur
    \item \underline{Acteur Principal} : Superviseur
    \item \underline{Parties prenantes et intérêts} :
    \begin{itemize}[label = \textbullet, font = \color{blue}]
        \item \underline{Le Superviseur} : il veut modifier ou supprimer une
            livraison
    \end{itemize}
    \item \underline{Préconditions} : le Superviseur est identifié et
        authentifié sur le système, la feuille de route est chargée sur le
        système, il existe au moins un fichier XML pour une demande de
        livraison et un fichier pour un plan géographique.
    \item \underline{Postconditions} : les modifications sont enregistrés en
        version texte et accessible pour le Livreur concerné
    \item \underline{Cas Nominal} :
    \begin{enumerate}
        \item Le Superviseur se retrouve sur la vue qui affiche les feuilles de
            routes.
        \item Dans le menu "Fichier" il clique sur "Charger un plan" ce qui
            correspond au cas d'utilisation "Charger un plan". Cela sera la
            seule option dans le menu qui est cliquable pour l'instant. Ceci
            correspond au CdU "Charger un plan", visible ci-dessous.
        \item Dans un deuxième temps, il va encore une fois dans le menu
            "Fichier". Maintenant l'option "Charger une demande" est activée
            aussi. Le Superviseur clique là-dessus pour charger une demande de
            livraisons ce qui correspond au CdU "Charger une demande", visible
            ci-dessous..
        \item La vue lui affiche maintenant deux listes de livraisons, la
            première composée de livraisons déjà inclues dans la tournée, la
            deuxième montrant la demande des livraisons pas encore traitées,
            ainsi qu'un plan de la zone précisée. En plus, les boutons
            "Supprimer" (au-dessous de la première liste), "Ajouter"
            (au-dessous de la deuxième liste) et "Calculer l'itinéraire"
            (au-dessous du plan) seront activées.
        \item Maintenant, le Superviseur peut cliquer sur une des livraisons
            dans n'importe quelle liste pour la faire dérouler et montrer les
            détails ou bien il peut cliquer sur un des points sur le plan pour
            faire pareil.
        \item S'il clique sur un point sur le plan qui est déjà inclu dans le
            chemin, le bouton "Ajouter" est désactivé, le bouton "Supprimer"
            reste cliquable.
        \item S'il choisit un point non-inclu, le bouton "Ajouter" reste activé
            et le bouton "Supprimer" devient gris et non-cliquable.
        \item Le Superviseur sélectionne une livraison de la première liste et
            clique sur le bouton "Supprimer". La livraison dite est viré de la
            tournée, et du chemin marqué sur le plan.
        \item Il choisit une livraison de la liste de demande, puis il clique
            sur le bouton "Ajouter". La livraison est viré de la demande de
            livraison mais elle se retrouve dans la tournée, donc la première
            liste.
        \item Une fois le superviseur satisfait de l'ordre des livraisons de la
            tournée, il clique sur le bouton "Calculer l'itinéraire", qui
            correspond au CdU "Calculer l'itinéraire", visible ci-dessous.
        \item Quand l'itinéraire est bien calculé, les livraisons de la tournée
            sont marquées d'un code couleur ainsi que les points de livraisons
            sur le plan:
        \begin{itemize}
            \item Blanc pour les livraisons qui entre dans la tournée sans aucun problème.
            \item Rouge pour celles qui ne peuvent pas être réalisées dans leurs créneaux ou bien qui sont en dehors du temps de travail.
        \end{itemize}
        \item En suite, le Superviseur peut recommencer soit à partir de point 2 soit à partir de point 3.
    \end{enumerate}
    \item \underline{Extensions} : Néant.
    \item \underline{Spécifications particulières} : Néant.
    \item \underline{Fréquence d'occurence} : potentiellement plusieurs fois par jour par feuille de route
    \item \underline{Divers} : Néant.
\end{itemize}

\subsection{Charger le plan}

\begin{itemize}[label = \textbullet, font = \color{orange}]
    \item \underline{Nom du cas d'utilisation} : Charger le plan
    \item \underline{Périmètre} : prototype OptiFret
    \item \underline{Niveau} : but utilisateur
    \item \underline{Acteur Principal} : Superviseur
    \item \underline{Parties prenantes et intérêts} :
    \begin{itemize}[label = \textbullet, font = \color{blue}]
        \item \underline{Le Superviseur} : il veut charger le plan
    \end{itemize}
    \item \underline{Préconditions} : le Superviseur est identifié et
        authentifié sur le système
    \item \underline{Postconditions} : la carte est chargée en mémoire (ou non,
        si l'action a été annulée)
    \item \underline{Cas Nominal} :
    \begin{enumerate}
        \item Le Superviseur se retrouve sur la vue qui affiche les feuilles de
            routes.
        \item Il sélectionne dans le menu "Fichier" l'action "Charger le plan".
        \item Une nouvelle fenêtre s'ouvre pour choisir un fichier XML. Tous
            les autres types de fichiers ne sonts pas choisissables.
        \item Il sélectionne un fichier XML correspondant à un plan et confirme
            son choix.
        \item Le fenêtre se ferme et il retourne sur la vue d'affichage.
    \end{enumerate}
    \item \underline{Extensions} :
    \begin{enumerate}
        \item Le Superviseur essaye de charger un fichier "invalide"
            \begin{enumerate}
                \item Le système refuse le chargement, et affiche un
                    message d'erreur ("Fichier invalide").
            \end{enumerate}
    \end{enumerate}
    \item \underline{Spécifications particulières} : Néant.
    \item \underline{Fréquence d'occurence} : potentiellement plusieurs fois
        par jour
    \item \underline{Divers} : Néant.
\end{itemize}

\subsection{Charger une demande de livraison}

\begin{itemize}[label = \textbullet, font = \color{orange}]
    \item \underline{Nom du cas d'utilisation} : Charger une demande de livraison
    \item \underline{Périmètre} : prototype OptiFret
    \item \underline{Niveau} : but utilisateur
    \item \underline{Acteur Principal} : Superviseur
    \item \underline{Parties prenantes et intérêts} :
    \begin{itemize}[label = \textbullet, font = \color{blue}]
        \item \underline{Le Superviseur} : il veut charger une demande de
            livraisons.
    \end{itemize}
    \item \underline{Préconditions} : le Superviseur est identifié et
        authentifié sur le système, le plan a été chargé.
    \item \underline{Postconditions} : les demandes de livraisons sont chargées
        en mémoire (ou non, si l'action a été annulée)
    \item \underline{Cas Nominal} :
    \begin{enumerate}
        \item Le Superviseur se retrouve sur la vue qui affiche les feuilles de
            routes.
        \item Il sélectionne l'action "Charger une demande de livraison" dans
            le menu "Fichier".
        \item Une nouvelle fenêtre pour choisir un fichier XML s'ouvre.
        \item Il sélectionne un fichier XML correspondant à une demande de
            livraison et confirme l'action, tous les autres types de fichiers
            ne sont pas choisissable.
        \item La fenêtre se ferme et il retourne sur la vue d'affichage.
    \end{enumerate}
    \item \underline{Extensions} :
    \begin{enumerate}
        \item Le Superviseur essaye de charger un fichier "invalide"
        \begin{enumerate}
            \item Le système refuse le chargement, et affiche un message
                d'erreur ("Fichier invalide").
        \end{enumerate}
    \end{enumerate}
    \item \underline{Spécifications particulières} : Néant.
    \item \underline{Fréquence d'occurence} : potentiellement plusieurs fois
        par jour
    \item \underline{Divers} : Néant.
\end{itemize}

\subsection{Calculer l'intinéraire}

\begin{itemize}[label = \textbullet, font = \color{orange}]
    \item \underline{Nom du cas d'utilisation} : Calculer l'itinéraire
    \item \underline{Périmètre} : prototype OptiFret
    \item \underline{Niveau} : but utilisateur
    \item \underline{Acteur Principal} : Superviseur
    \item \underline{Parties prenantes et intérêts} :
    \begin{itemize}[label = \textbullet, font = \color{blue}]
        \item \underline{Le Superviseur} : il veut calculer le chemin le plus
            court entre les livraisons d'une tournée
    \end{itemize}
    \item \underline{Préconditions} : Le plan d'une zone géographique a été
        chargé ainsi que la demande de livraisons.
    \item \underline{Postconditions} : un chemin a été calculé et les
        livraisons compris dans la tournée sont marquées d'après le code
        couleur spécifié dans le CdU "Modifier une feuille de route avant le
        début de livraison"
    \item \underline{Cas Nominal} :
    \begin{enumerate}
        \item Le Superviseur se trouve sur la vue de la modification d'une
            tournée avant qu'elle ait été commencé et il a déjà charger un plan
            et une demande de livraison.
        \item Il clique sur le bouton "Calculer l'itinéraire".
        \item Le système lui affiche une barre d'attente pendant le traitement
            des données.
        \item Une fois le calcul terminé, la barre d'attente se ferme et le
            chemin calculé est dessiné sur le plan.
        \item Les livraisons dans la tournée sont colorées ainsi que les points
            correspondants et les tronçons sur le plan.
    \end{enumerate}
    \item \underline{Extensions} : Néant.
    \item \underline{Spécifications particulières} : Néant.
    \item \underline{Fréquence d'occurence} : Potentiellement plusieurs fois
        pendant la modification d'une feuille de route.
    \item \underline{Divers} : Néant.
\end{itemize}

\section{Diagrammes de séquence}
\subsection{Charger les demandes de livraison et calculer la feuille de route}

\begingroup
\fontsize{4pt}{5pt}\selectfont
\centering
\begin{sequencediagram}
    \newthread{app}{OptiFret}
    \newinst{lrnListener}{lrnListener: MouseListener}
    \newinst{ldsListener}{ldsListener: MouseListener}
    \newinst{dsModel}{:DeliverySheet}
    \newinst{rnModel}{:RoadNetwork}
    \newinst{controller}{:Controller}
    \newinst{view}{:View}

    \begin{sdblock}{}{}
        \begin{call}{app}{activate()}{lrnListener}{}
            \begin{call}{lrnListener}{loadRoadNetwork()}{controller}{}
                \begin{sdblock}{}{}
                    \begin{callself}{controller}{selectFile()}{file}
                    \end{callself}
                    \begin{call}{controller}{loadFromXML(file)}{rnModel}{}
                    \end{call}
                    \begin{call}{controller}{update()}{view}{}
                    \end{call}
                \end{sdblock}
            \end{call}
        \end{call}
        \begin{call}{app}{activate()}{ldsListener}{}
            \begin{call}{ldsListener}{loadDeliverySheet()}{controller}{}
                \begin{sdblock}{}{}
                    \begin{callself}{controller}{selectFile()}{file}
                    \end{callself}
                    \begin{call}{controller}{loadFromXML(file)}{dsModel}{}
                    \end{call}
                    \begin{call}{controller}{getDeliveryNodes()}{dsModel}{nodes}
                    \end{call}
                    \begin{call}{controller}{makeRoute(nodes)}{rnModel}{route}
                    \end{call}
                    \begin{call}{controller}{setNodes(route)}{dsModel}{}
                    \end{call}
                    \begin{call}{controller}{update()}{view}{}
                    \end{call}
                \end{sdblock}
            \end{call}
        \end{call}
    \end{sdblock}
\end{sequencediagram}
\endgroup

\subsection{Modifier interactivement la feuille de route}
{\huge TODO}

\end{document}
